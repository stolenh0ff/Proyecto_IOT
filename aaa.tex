\documentclass[12pt,letterpaper]{article}

\input{../Commons/preamble}
\usepackage{hyperref}

% Ingrese los datos de la asignatura

\newcommand{\classCode}{SIY6122}
\newcommand{\className}{Problematicas Globales Y Prototipado}
\newcommand{\classSemester}{2022-2}
\newcommand{\classParallel}{001D}

% Ingrese el título del informe

\newcommand{\reportTitle}{Informe de Proyecto}

\newcommand{\groupStudentARol}{20.434.624-0}
\newcommand{\groupStudentAName}{Benjamín Sepúlveda}
\newcommand{\groupStudentAEmail}{benj.sepulvedas@duocuc.cl}

\newcommand{\groupStudentBRol}{20.881.648-9}
\newcommand{\groupStudentBName}{Alonso Guajardo}
\newcommand{\groupStudentBEmail}{alon.guajardo@duocuc.cl}

\newcommand{\groupStudentCRol}{19.841.277-5}
\newcommand{\groupStudentCName}{Matías Toro}
\newcommand{\groupStudentCEmail}{mat.torot@duocuc.cl}

\newcommand{\groupStudentDRol}{20.755.578-9}
\newcommand{\groupStudentDName}{Andrés Ávalos}
\newcommand{\groupStudentDEmail}{correo@duocuc.cl}

\newcommand{\groupStudentERol}{rut}
\newcommand{\groupStudentEName}{Alex Sanhueza}
\newcommand{\groupStudentEEmail}{correo@duocuc.cl}

\begin{document}
\input{../Commons/initialpages}

\section{Introducción}

En el siguiente informe se detallará a cavalidad

Aqui irá la Introducción

\newpage

\section{Desarrollo}




Desarrollo del Informe

\subsection{Objetivo Nr1}

•	Describa tres características de soluciones AAA que permitan la mejora de los sistemas de autenticación.

\begin{itemize}
    \item Característica 1: Mayor capacidad de flexibilidad y control de la configuración de acceso.
    \item Característica 2: Garantizar que el acceso a los recursos de la red y de las aplicaciones de software.
    \item Característica 3: Uso de múltiples dispositivos de copia de seguridad.
\end{itemize}

\begin{figure}[H]
    \centering
    \captionsetup{width=.4\linewidth}
    \includegraphics[scale=0.35]{Commons/images/linux/step1.png}
    \caption{Creación de Archivo de Texto}
    \label{fig:ipc}
\end{figure}

\subsection{Objetivo Nr2}
\subsection{Objetivo Nr3}
\subsection{Objetivo Nr4}

\newpage
\section{Conclusión}

TEXTO CONCLUSIÓN

\newpage

\section{Referencias Bibliográficas}
\begin{thebibliography}{0}
    

https://blog.oxfamintermon.org/por-que-hacer-un-uso-responsable-del-agua/#:~:text=Nada%20de%20afeitarse%2C%20lavarse%20las,hayan%20ca%C3%ADdo%20desde%20el%20exterior.

https://www.fundacionaquae.org/sabes-cuanta-agua-consumes-a-diario/

https://www.un.org/es/chronicle/article/como-podemos-reducir-nuestra-huella-de-agua-un-nivel-sostenible

https://tecnicrop.com/blog/optimizacion-del-uso-del-agua-en-agricultura

https://www.greenpeace.org/mexico/noticia/9460/como-afecta-el-cambio-climatico-el-acceso-al-agua/

https://www.fundacionaquae.org/agua-cambio-climatico-efectos/#:~:text=El%20cambio%20clim%C3%A1tico%20se%20manifiesta,se%20intensifican%20con%20graves%20consecuencias.

https://www.iagua.es/blogs/ricardo-perez/agua-agricultura-importancia-y-manejo#:~:text=El%20uso%20de%20agua%20agr%C3%ADcola,ligero)%20y%20control%20de%20heladas.

    
    \bibitem{Universidad de Chile}
    \textit{uchile}. (Marzo 2018). La necesidad de reducir el consumo de agua ante el inminente agotamiento de los recursos hídricos
    \url{https://www.uchile.cl/noticias/141937/la-necesidad-de-reducir-el-consumo-de-agua-ante-su-agotamiento}
    
    
    \bibitem{}
    \textit{}. (Mayo 2021). titulo
    \url{}
        

    \bibitem{IDS/IPS}
    \textit{Geekflare}. (Agosto 2021). IDS vs IPS: una guía completa de soluciones de seguridad de red
    \url{https://geekflare.com/es/ids-vs-ips-network-security-solutions/}
    
        
\end{thebibliography}

\end{document}